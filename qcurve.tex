% === example 1, poly4_lv2 ===
\newcommand{\qcurvePoly}{

\begin{figure}[]
\centering
  \begin{subfigure}[h]{0.5\textwidth}
    \includegraphics[width=\textwidth]{./fig/poly4_lv2/quantileIGP_MP.eps}
    \caption{IGP+MP}
    %\label{fig:gull}
  \end{subfigure}%
  ~
  \begin{subfigure}[h]{0.5\textwidth}
    \includegraphics[width=\textwidth]{./fig/poly4_lv2/quantileIGP_EI.eps}
    \caption{IGP+MSEI}
    %\label{fig:gull}
  \end{subfigure}%
  
  \begin{subfigure}[h]{0.5\textwidth}
    \includegraphics[width=\textwidth]{./fig/poly4_lv2/quantileQQGP_MP.eps}
    \caption{QQGP+MP}
    %\label{fig:gull}
  \end{subfigure}%
  ~
  \begin{subfigure}[h]{0.5\textwidth}
    \includegraphics[width=\textwidth]{./fig/poly4_lv2/quantileQQGP_EI.eps}
    \caption{QQGP+MSEI}
    %\label{fig:poly}
  \end{subfigure}%
  \caption{Quantile curve of poly4\_lv2 }\label{fig:poly4_lv2_qcurve}
\end{figure}

}
% =============================

% === example 2, spline_lv2 ===
\newcommand{\qcurveSplineLow}{

\begin{figure}[]
\centering
  \begin{subfigure}[h]{0.5\textwidth}
    \includegraphics[width=\textwidth]{./fig/spline_lv2/quantileIGP_MP.eps}
    \caption{IGP+MP}
    %\label{fig:gull}
  \end{subfigure}%
  ~
  \begin{subfigure}[h]{0.5\textwidth}
    \includegraphics[width=\textwidth]{./fig/spline_lv2/quantileIGP_EI.eps}
    \caption{IGP+MSEI}
    %\label{fig:gull}
  \end{subfigure}%
  
  \begin{subfigure}[h]{0.5\textwidth}
    \includegraphics[width=\textwidth]{./fig/spline_lv2/quantileQQGP_MP.eps}
    \caption{QQGP+MP}
    %\label{fig:gull}
  \end{subfigure}%
  ~
  \begin{subfigure}[h]{0.5\textwidth}
    \includegraphics[width=\textwidth]{./fig/spline_lv2/quantileQQGP_EI.eps}
    \caption{QQGP+MSEI}
    %\label{fig:poly}
  \end{subfigure}%
  \caption{Quantile curve of spline\_lv2 }\label{fig:spline_lv2_qcurve}
\end{figure}

}
% =============================

% === example 3, spline_lv5 ===
\newcommand{\qcurveSplineHigh}{

\begin{figure}[]
\centering
  \begin{subfigure}[h]{0.5\textwidth}
    \includegraphics[width=\textwidth]{./fig/spline_lv5/quantileIGP_MP.eps}
    \caption{IGP+MP}
    %\label{fig:gull}
  \end{subfigure}%
  ~
  \begin{subfigure}[h]{0.5\textwidth}
    \includegraphics[width=\textwidth]{./fig/spline_lv5/quantileIGP_EI.eps}
    \caption{IGP+MSEI}
    %\label{fig:gull}
  \end{subfigure}%
  
  \begin{subfigure}[h]{0.5\textwidth}
    \includegraphics[width=\textwidth]{./fig/spline_lv5/quantileQQGP_MP.eps}
    \caption{QQGP+MP}
    %\label{fig:gull}
  \end{subfigure}%
  ~
  \begin{subfigure}[h]{0.5\textwidth}
    \includegraphics[width=\textwidth]{./fig/spline_lv5/quantileQQGP_EI.eps}
    \caption{QQGP+MSEI}
    %\label{fig:poly}
  \end{subfigure}%
  \caption{Quantile curve of spline\_lv5 }\label{fig:spline_lv5_qcurve}
\end{figure}

}
% =============================

% === example 4, gabor_lv3 ===
\newcommand{\qcurveGabor}{

\begin{figure}[]
\centering
  \begin{subfigure}[h]{0.5\textwidth}
    \includegraphics[width=\textwidth]{./fig/gabor_lv3/quantileIGP_MP.eps}
    \caption{IGP+MP}
    %\label{fig:gull}
  \end{subfigure}%
  ~
  \begin{subfigure}[h]{0.5\textwidth}
    \includegraphics[width=\textwidth]{./fig/gabor_lv3/quantileIGP_EI.eps}
    \caption{IGP+MSEI}
    %\label{fig:gull}
  \end{subfigure}%
  
  \begin{subfigure}[h]{0.5\textwidth}
    \includegraphics[width=\textwidth]{./fig/gabor_lv3/quantileQQGP_MP.eps}
    \caption{QQGP+MP}
    %\label{fig:gull}
  \end{subfigure}%
  ~
  \begin{subfigure}[h]{0.5\textwidth}
    \includegraphics[width=\textwidth]{./fig/gabor_lv3/quantileQQGP_EI.eps}
    \caption{QQGP+MSEI}
    %\label{fig:poly}
  \end{subfigure}%
  \caption{Quantile curve of gabor\_lv3 }\label{fig:gabor_lv3_qcurve}
\end{figure}

}
% =============================

% === example 5(a), amg_ani_cg ===
\newcommand{\qcurveAmgAniCG}{

\begin{figure}[]
\centering
  \begin{subfigure}[h]{0.5\textwidth}
    \includegraphics[width=\textwidth]{./fig/amg_ani_cg/quantileIGP_MP.eps}
    \caption{IGP+MP}
    %\label{fig:gull}
  \end{subfigure}%
  ~
  \begin{subfigure}[h]{0.5\textwidth}
    \includegraphics[width=\textwidth]{./fig/amg_ani_cg/quantileIGP_EI.eps}
    \caption{IGP+MSEI}
    %\label{fig:gull}
  \end{subfigure}%
  
  \begin{subfigure}[h]{0.5\textwidth}
    \includegraphics[width=\textwidth]{./fig/amg_ani_cg/quantileQQGP_MP.eps}
    \caption{QQGP+MP}
    %\label{fig:gull}
  \end{subfigure}%
  ~
  \begin{subfigure}[h]{0.5\textwidth}
    \includegraphics[width=\textwidth]{./fig/amg_ani_cg/quantileQQGP_EI.eps}
    \caption{QQGP+MSEI}
    %\label{fig:poly}
  \end{subfigure}%
  \caption{Quantile curve of amg\_ani\_cg }\label{fig:amg_ani_cg_qcurve}
\end{figure}

}
% =============================

% === example 5(b), amg_ani_bicgstab ===
\newcommand{\qcurveAmgAniBICGSTAB}{

\begin{figure}[]
\centering
  \begin{subfigure}[h]{0.5\textwidth}
    \includegraphics[width=\textwidth]{./fig/amg_ani_bicgstab/quantileIGP_MP.eps}
    \caption{IGP+MP}
    %\label{fig:gull}
  \end{subfigure}%
  ~
  \begin{subfigure}[h]{0.5\textwidth}
    \includegraphics[width=\textwidth]{./fig/amg_ani_bicgstab/quantileIGP_EI.eps}
    \caption{IGP+MSEI}
    %\label{fig:gull}
  \end{subfigure}%
  
  \begin{subfigure}[h]{0.5\textwidth}
    \includegraphics[width=\textwidth]{./fig/amg_ani_bicgstab/quantileQQGP_MP.eps}
    \caption{QQGP+MP}
    %\label{fig:gull}
  \end{subfigure}%
  ~
  \begin{subfigure}[h]{0.5\textwidth}
    \includegraphics[width=\textwidth]{./fig/amg_ani_bicgstab/quantileQQGP_EI.eps}
    \caption{QQGP+MSEI}
    %\label{fig:poly}
  \end{subfigure}%
  \caption{Quantile curve of amg\_ani\_bicgstab }\label{fig:amg_ani_bicgstab_qcurve}
\end{figure}

}
% =============================

% === example 5(c), amg_iso_cg ===
\newcommand{\qcurveAmgIsoCG}{

\begin{figure}[]
\centering
  \begin{subfigure}[h]{0.5\textwidth}
    \includegraphics[width=\textwidth]{./fig/amg_iso_cg/quantileIGP_MP.eps}
    \caption{IGP+MP}
    %\label{fig:gull}
  \end{subfigure}%
  ~
  \begin{subfigure}[h]{0.5\textwidth}
    \includegraphics[width=\textwidth]{./fig/amg_iso_cg/quantileIGP_EI.eps}
    \caption{IGP+MSEI}
    %\label{fig:gull}
  \end{subfigure}%
  
  \begin{subfigure}[h]{0.5\textwidth}
    \includegraphics[width=\textwidth]{./fig/amg_iso_cg/quantileQQGP_MP.eps}
    \caption{QQGP+MP}
    %\label{fig:gull}
  \end{subfigure}%
  ~
  \begin{subfigure}[h]{0.5\textwidth}
    \includegraphics[width=\textwidth]{./fig/amg_iso_cg/quantileQQGP_EI.eps}
    \caption{QQGP+MSEI}
    %\label{fig:poly}
  \end{subfigure}%
  \caption{Quantile curve of amg\_iso\_cg }\label{fig:amg_iso_cg_qcurve}
\end{figure}

}
% =============================

% === example 5(d), amg_iso_bicgstab ===
\newcommand{\qcurveAmgIsoBICGSTAB}{

\begin{figure}[]
\centering
  \begin{subfigure}[h]{0.5\textwidth}
    \includegraphics[width=\textwidth]{./fig/amg_iso_bicgstab/quantileIGP_MP.eps}
    \caption{IGP+MP}
    %\label{fig:gull}
  \end{subfigure}%
  ~
  \begin{subfigure}[h]{0.5\textwidth}
    \includegraphics[width=\textwidth]{./fig/amg_iso_bicgstab/quantileIGP_EI.eps}
    \caption{IGP+MSEI}
    %\label{fig:gull}
  \end{subfigure}%
  
  \begin{subfigure}[h]{0.5\textwidth}
    \includegraphics[width=\textwidth]{./fig/amg_iso_bicgstab/quantileQQGP_MP.eps}
    \caption{QQGP+MP}
    %\label{fig:gull}
  \end{subfigure}%
  ~
  \begin{subfigure}[h]{0.5\textwidth}
    \includegraphics[width=\textwidth]{./fig/amg_iso_bicgstab/quantileQQGP_EI.eps}
    \caption{QQGP+MSEI}
    %\label{fig:poly}
  \end{subfigure}%
  \caption{Quantile curve of amg\_iso\_bicgstab }\label{fig:amg_iso_bicgstab_qcurve}
\end{figure}

}
% =============================

% === example 6(a), sim_ani_cg ===
\newcommand{\qcurveSimAniCG}{

\begin{figure}[]
\centering
  \begin{subfigure}[h]{0.5\textwidth}
    \includegraphics[width=\textwidth]{./fig/sim_ani_cg/quantileIGP_MP.eps}
    \caption{IGP+MP}
    %\label{fig:gull}
  \end{subfigure}%
  ~
  \begin{subfigure}[h]{0.5\textwidth}
    \includegraphics[width=\textwidth]{./fig/sim_ani_cg/quantileIGP_EI.eps}
    \caption{IGP+MSEI}
    %\label{fig:gull}
  \end{subfigure}%
  
  \begin{subfigure}[h]{0.5\textwidth}
    \includegraphics[width=\textwidth]{./fig/sim_ani_cg/quantileQQGP_MP.eps}
    \caption{QQGP+MP}
    %\label{fig:gull}
  \end{subfigure}%
  ~
  \begin{subfigure}[h]{0.5\textwidth}
    \includegraphics[width=\textwidth]{./fig/sim_ani_cg/quantileQQGP_EI.eps}
    \caption{QQGP+MSEI}
    %\label{fig:poly}
  \end{subfigure}%
  \caption{Quantile curve of sim\_ani\_cg }\label{fig:sim_ani_cg_qcurve}
\end{figure}

}
% =============================

% === example 6(b), sim_ani_bicgstab ===
\newcommand{\qcurveSimAniBICGSTAB}{

\begin{figure}[]
\centering
  \begin{subfigure}[h]{0.5\textwidth}
    \includegraphics[width=\textwidth]{./fig/sim_ani_bicgstab/quantileIGP_MP.eps}
    \caption{IGP+MP}
    %\label{fig:gull}
  \end{subfigure}%
  ~
  \begin{subfigure}[h]{0.5\textwidth}
    \includegraphics[width=\textwidth]{./fig/sim_ani_bicgstab/quantileIGP_EI.eps}
    \caption{IGP+MSEI}
    %\label{fig:gull}
  \end{subfigure}%
  
  \begin{subfigure}[h]{0.5\textwidth}
    \includegraphics[width=\textwidth]{./fig/sim_ani_bicgstab/quantileQQGP_MP.eps}
    \caption{QQGP+MP}
    %\label{fig:gull}
  \end{subfigure}%
  ~
  \begin{subfigure}[h]{0.5\textwidth}
    \includegraphics[width=\textwidth]{./fig/sim_ani_bicgstab/quantileQQGP_EI.eps}
    \caption{QQGP+MSEI}
    %\label{fig:poly}
  \end{subfigure}%
  \caption{Quantile curve of sim\_ani\_bicgstab }\label{fig:sim_ani_bicgstab_qcurve}
\end{figure}

}
% =============================

% === example 6(c), sim_iso_cg ===
\newcommand{\qcurveSimIsoCG}{

\begin{figure}[]
\centering
  \begin{subfigure}[h]{0.5\textwidth}
    \includegraphics[width=\textwidth]{./fig/sim_iso_cg/quantileIGP_MP.eps}
    \caption{IGP+MP}
    %\label{fig:gull}
  \end{subfigure}%
  ~
  \begin{subfigure}[h]{0.5\textwidth}
    \includegraphics[width=\textwidth]{./fig/sim_iso_cg/quantileIGP_EI.eps}
    \caption{IGP+MSEI}
    %\label{fig:gull}
  \end{subfigure}%
  
  \begin{subfigure}[h]{0.5\textwidth}
    \includegraphics[width=\textwidth]{./fig/sim_iso_cg/quantileQQGP_MP.eps}
    \caption{QQGP+MP}
    %\label{fig:gull}
  \end{subfigure}%
  ~
  \begin{subfigure}[h]{0.5\textwidth}
    \includegraphics[width=\textwidth]{./fig/sim_iso_cg/quantileQQGP_EI.eps}
    \caption{QQGP+MSEI}
    %\label{fig:poly}
  \end{subfigure}%
  \caption{Quantile curve of sim\_iso\_cg }\label{fig:sim_iso_cg_qcurve}
\end{figure}

}
% =============================

% === example 6(d), amg_iso_bicgstab ===
\newcommand{\qcurveSimIsoBICGSTAB}{

\begin{figure}[]
\centering
  \begin{subfigure}[h]{0.5\textwidth}
    \includegraphics[width=\textwidth]{./fig/sim_iso_bicgstab/quantileIGP_MP.eps}
    \caption{IGP+MP}
    %\label{fig:gull}
  \end{subfigure}%
  ~
  \begin{subfigure}[h]{0.5\textwidth}
    \includegraphics[width=\textwidth]{./fig/sim_iso_bicgstab/quantileIGP_EI.eps}
    \caption{IGP+MSEI}
    %\label{fig:gull}
  \end{subfigure}%
  
  \begin{subfigure}[h]{0.5\textwidth}
    \includegraphics[width=\textwidth]{./fig/sim_iso_bicgstab/quantileQQGP_MP.eps}
    \caption{QQGP+MP}
    %\label{fig:gull}
  \end{subfigure}%
  ~
  \begin{subfigure}[h]{0.5\textwidth}
    \includegraphics[width=\textwidth]{./fig/sim_iso_bicgstab/quantileQQGP_EI.eps}
    \caption{QQGP+MSEI}
    %\label{fig:poly}
  \end{subfigure}%
  \caption{Quantile curve of sim\_iso\_bicgstab }\label{fig:sim_iso_bicgstab_qcurve}
\end{figure}

}
% =============================

\newcommand{\qcurveAll}{
\qcurvePoly
\qcurveSplineLow
\qcurveSplineHigh
\qcurveGabor
\qcurveAmgAniCG
\qcurveAmgAniBICGSTAB
\qcurveAmgIsoCG
\qcurveAmgIsoBICGSTAB
\qcurveSimAniCG
\qcurveSimAniBICGSTAB
\qcurveSimIsoCG
\qcurveSimIsoBICGSTAB
}
